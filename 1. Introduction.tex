%%
% Please see https://bitbucket.org/rivanvx/beamer/wiki/Home for obtaining beamer.
%%
\documentclass[aspectratio=169]{beamer}
\usetheme{Antibes}
\usepackage{xcolor}
\mode<presentation>

\useoutertheme{miniframes} 
\useinnertheme{circles}

\definecolor{primary}{HTML}{003469}
\definecolor{secondary}{HTML}{003469}
\definecolor{tertiary}{HTML}{00addc}

\setbeamercolor{titlelike}{bg=white,fg=primary}

\setbeamercolor{palette primary}{bg=tertiary,fg=white}
\setbeamercolor{palette secondary}{bg=tertiary,fg=white}
\setbeamercolor{palette tertiary}{bg=secondary,fg=white}
\setbeamercolor{structure}{fg=secondary} % itemize, enumerate, etc
\setbeamercolor{section in toc}{fg=secondary} % TOC sections
% Override palette coloring with secondary
\setbeamercolor{subsection in head/foot}{bg=tertiary,fg=white}
\usepackage{natbib}
\bibliographystyle{unsrtnat}
\setcitestyle{authoryear,open={(},close={)}}
\usepackage{csquotes}


\title{\large{\textbf{Python for TE}} \newline\newline 1. Introduction}

\author{Jack Minchin}
\institute{Tourism Economics}
\date{2022}

\begin{document}

\frame{\titlepage}

\begin{frame}
\frametitle{Table of Contents}
\tableofcontents
\end{frame}

\section{Introduction}

\begin{frame}{Outline}
	\begin{block}{Disclaimer}
		Probably contains over-simplications and some errors in an attempt to simplify.
	\end{block}

	\begin{enumerate}
		\item Part 1:
		\begin{itemize}
			\item What is Python?
			\item How can it be used?
			\item How can I learn?
		\end{itemize}

	\end{enumerate}
\end{frame}


\begin{frame}{What Python is not}
\framesubtitle{Some common misconceptions}

\begin{itemize}
	\item A statistical package
	\item A program
	\item A self contained framework
\end{itemize}

\end{frame}


\begin{frame}{What is Python?}
\blockquote[wikipedia]{
	Python is a high-level, general-purpose programming language. Its design philosophy emphasizes code readability with the use of significant indentation. Its language constructs and object-oriented approach aim to help programmers write clear, logical code for small- and large-scale projects.}
\end{frame}

\section{Some Use-Cases}

\begin{frame}{How can we use it?}

\begin{enumerate}
	\item \textbf{Extraction, Transform and Load (ETL) Tasks}

			Reading data from (virtually) any source, transforming into required format and exporting.
	
	\item \textbf{Automation of repetitive tasks.}
	
			Interacting with file system, using libraries to create PowerPoints, interacting directly with excel, read emails - the possibilities really are endless.
			
	\item \textbf{Data Visualisation, Analysis and Econometrics}
	
	\item \textbf{Full scale applications}
		
			Larger projects will span multiple categories (e.g Country Profile Report automation)\footnote{Current implementionation uses another language called JavaScript (Node.js) for the PDF generation.}
	
\end{enumerate}
	
\end{frame}

\section{Resources for learning}

\begin{frame}{How to learn Python}

\begin{enumerate}

\item \textbf{I can't teach you.}
\item \textbf{Learning-by-doing}

		The \textbf{best} way to learn Python is to start writing it. There are multiple courses online, from free YouTube videos to paid courses but in-person courses will not help unless you are spending hours working alone figuring out.
		
\item \textbf{DataCamp.com}
	
		Combines very short videos with active excercices, it teaches concepts in a concise way and then forces you to get involved. It is also designed for ETL and statistics. 
	
\end{enumerate}
\end{frame}

\begin{frame}{The Coding Learning Curve}


\centering
\includegraphics[width=0.7\linewidth]{graphics/complex-lc.png}
	
\end{frame}






\end{document}
